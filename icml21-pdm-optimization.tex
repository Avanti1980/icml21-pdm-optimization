\documentclass{ctexart}
\usepackage{avanti}

\begin{document}
\title{PDM的优化}
\author{Teng Zhang}
\date{\today}
\maketitle

目前的优化问题为
\begin{align*}
    \min_{\wv} ~ \frac{1}{2m} \sum_{i \in [m]} \max \left\{ 0, 1 - \frac{y_i \wv^\top \xv_i}{\bar{\gamma}} \right\}^2, \quad \st ~ \bar{\gamma} > 0, ~ \| \wv \|_2 = 1
\end{align*}
这个形式是有问题的,两个约束都是多余的。注意间隔均值$\bar{\gamma} = \wv^\top \overline{y \xv}$关于$\wv$也是线性的,所以那个比值项是齐次的,放缩$\wv$不影响目标函数值。若求得的$\wv$使得$\bar{\gamma} < 0$,则$-\wv$可使得$\bar{\gamma} > 0$且不改变目标函数值、不破坏第二个约束。同理第二个约束也是多余的,若求得的$\wv$范数不为$1$,可以将其缩放使范数为$1$且不改变目标函数值。

\section{重新优化}

将两个约束都去掉,问题变成
\begin{align*}
    \min_{\wv} ~ \frac{1}{2m} \sum_{i \in [m]} \max \left\{ 0, 1 - \frac{y_i \wv^\top \xv_i}{\bar{\gamma}} \right\}^2
\end{align*}
由于放缩$\wv$不影响目标函数值,不妨放缩使$\bar{\gamma} = 1$,于是问题可进一步写成
\begin{align*}
    \min_{\wv} ~ \frac{1}{2m} \sum_{i \in [m]} \max \left\{ 0, 1 - y_i \wv^\top \xv_i \right\}^2, \quad \st ~ \sum_{i \in [m]} y_i \wv^\top \xv_i = m
\end{align*}
这个形式直觉上也是合理的,最优情况是所有样本间隔均为$1$,都等于间隔均值,这样损失为零;若有样本间隔$> 1$,由于间隔的和为$m$,必然存在另一个样本间隔$< 1$,这样就会产生损失。

记$\epsilon_i = \max \{ 0, 1 - y_i \wv^\top \xv_i \}$,则问题可写成
\begin{align*}
    \min_{\wv, \epsilon_i} ~ \frac{1}{2m} \sum_{i \in [m]} \epsilon_i^2, \quad \st ~ \epsilon_i \ge 0, ~ \epsilon_i \ge 1 - y_i \wv^\top \xv_i, ~ \sum_{i \in [m]} y_i \wv^\top \xv_i = m
\end{align*}
其中约束$\epsilon_i \ge 0$是多余的,因为若求得的$\epsilon_i < 0$,令其为零不破坏约束$\epsilon_i \ge 1 - y_i \wv^\top \xv_i$且可以使目标函数值变小,因此不存在这种情况。最终的问题为
\begin{align} \label{eq: primal}
    \min_{\wv, \epsilon_i} ~ \frac{1}{2m} \sum_{i \in [m]} \epsilon_i^2, \quad \st ~ \epsilon_i \ge 1 - y_i \wv^\top \xv_i, ~ \sum_{i \in [m]} y_i \wv^\top \xv_i = m
\end{align}
目标函数是二次的,约束都是线性的,因此这是一个凸二次规划,直接调用成熟的QP优化包即可。

记$\Xv = [\xv_1, \xv_2, \ldots, \xv_m] \in \Rbb^{d \times m}$,$\Yv = \diag \{ y_1, y_2, \ldots, y_m\}$,$\epsilonv = [\epsilon_1; \epsilon_2; \ldots; \epsilon_m]$,式(\ref{eq: primal})的矩阵形式为
\begin{align*}
    \min_{\wv, \epsilonv} ~ \frac{1}{2m} \epsilonv^\top \epsilonv, \quad \st ~ \epsilonv \ge \ev - \Yv \Xv^\top \wv, ~ \ev^\top \Yv \Xv^\top \wv = m
\end{align*}
记$\uv = [\wv; \epsilonv] \in \Rbb^{d+m}$,写成QP的标准形式是
\begin{align*}
    \min_{\uv} ~ \frac{1}{2} \uv^\top \begin{bmatrix}
        \zerov \\ & \frac{\Iv}{m}
    \end{bmatrix} \uv, \quad \st ~ [- \Yv \Xv^\top, -\Iv] \uv \le -\ev, ~ [\Xv \Yv \ev; \zerov]^\top \uv = m
\end{align*}
这样就可以直接调包了。

%对偶函数为
%\begin{align*}
%    L(\epsilonv, \wv, \betav, \mu) = \frac{1}{2m} \epsilonv^\top \epsilonv - \betav^\top %(\epsilonv - \ev + \Yv \Xv^\top \wv) - \mu (\ev^\top \Yv \Xv^\top \wv - m)
%\end{align*}
%令$L$关于$\wv$的偏导数为零
%\begin{align*}
%    \frac{\partial L}{\partial \wv} = - \Xv \Yv \betav - \mu \Xv \Yv \ev = \zerov %\Longrightarrow \Xv \Yv \betav + \mu \Xv \Yv \ev = \zerov
%\end{align*}
%令$L$关于$\epsilonv$的偏导数为零
%\begin{align*}
%    \frac{\partial L}{\partial \epsilonv} = \frac{\epsilonv}{m} - \betav = \zerov %\Longrightarrow \epsilonv = m \betav
%\end{align*}
%故对偶问题为
%\begin{align*}
%    \max_{\betav, \mu} ~ - \frac{m}{2} \betav^\top \betav + \betav^\top \ev + \mu m, %\quad \st ~ \Xv \Yv \betav + \mu \Xv \Yv \ev = \zerov, ~ \betav \ge \zerov
%\end{align*}
%记$\alphav = [\betav; \mu] \in \Rbb^{m+1}$,写成QP的标准形式是
%\begin{align*}
%    \min_{\alphav} ~ \frac{1}{2} \alphav^\top \begin{bmatrix}
%        m \Iv \\ & 0
%    \end{bmatrix} \alphav - [\ev; m]^\top \alphav, \quad \st ~ [\Xv \Yv, \Xv \Yv \ev] %\alphav = \zerov, ~ \begin{bmatrix}
%        - \Iv \\ & 0
%    \end{bmatrix} \alphav \le \zerov
%\end{align*}
%
%原问题的变量个数为$d + m$,有$m$个线性不等式和$1$个线性等式约束;对偶问题的变量个数为$m+1$,有$d$个线性等式约束,不等式约束很简单可忽略,因此直接调包求解的话,可能对偶问题效率会稍微高点,但不绝对,跟$m$和$d$的值有关系。此外,由于两个问题都有大量形式复杂的约束,所以目前看不存在基于坐标下降的高效解法。

\section{核化}

设$\kappa (\xv, \zv) = \phi(\xv)^\top \phi(\zv)$,记$\Xv = [\phi(\xv_1), \phi(\xv_2), \ldots, \phi(\xv_m)]$,$\wv$的正交分解为
\begin{align*}
    \wv = \underbrace{a_1 \phi(\xv_1) + a_2 \phi(\xv_2) + \cdots + a_m \phi(\xv_m)}_{\phiv_{\xv}} + \phiv_{\xv}^\perp = \phiv_{\xv} + \phiv_{\xv}^\perp
\end{align*}
其中$\phiv_{\xv}^\perp \in \ker (\Xv^\top)$。注意$\wv$在原问题中只以$\Xv^\top \wv$的形式出现,又
\begin{align*}
    \Xv^\top \wv = \Xv^\top (\phiv_{\xv} + \phiv_{\xv}^\perp) = \Xv^\top \phiv_{\xv}
\end{align*}
\blue{故原问题最优的$\wv$不是唯一的},$\phiv_{\xv}^\perp$可以取$\ker (\Xv^\top)$中的任意向量。传统的核方法目标函数中还有$\wv$的二次正则,所以最小化
\begin{align*}
    \| \wv \|_2^2 = \| \phiv_{\xv} \|^2 + \| \phiv_{\xv}^\perp \|^2 \ge \| \phiv_{\xv} \|^2
\end{align*}
必然导致$\phiv_{\xv}^\perp = \zerov$,从而有
\begin{align*}
    \wv = a_1 \phi(\xv_1) + a_2 \phi(\xv_2) + \cdots + a_m \phi(\xv_m) = \Xv \av
\end{align*}
这就是所谓的表示定理。

对于PDM,没有二次正则,我们就直接取$\phiv_{\xv}^\perp = \zerov$好了,即$\wv = \Xv \av$(注意$\tilde{\wv} = \wv + \ker (\Xv^\top)$其实都是原问题的最优解),代入原问题有
\begin{align*}
    \min_{\av, \epsilonv} ~ \frac{1}{2m} \epsilonv^\top \epsilonv, \quad \st ~ \epsilonv \ge \ev - \Yv \Xv^\top \Xv \av, ~ \ev^\top \Yv \Xv^\top \Xv \av = m
\end{align*}
记$\Kv = \Xv^\top \Xv$为核矩阵,$\vv = [\av; \epsilonv] \in \Rbb^{2m}$,写成QP的标准形式是
\begin{align*}
    \min_{\vv} ~ \frac{1}{2} \vv^\top \begin{bmatrix}
        \zerov \\ & \frac{\Iv}{m}
    \end{bmatrix} \vv, \quad \st ~ [- \Yv \Kv, -\Iv] \vv \le -\ev, ~ [\Kv \Yv \ev; \zerov]^\top \vv = m
\end{align*}
解出$\vv$后,对于未知样本$\zv$,预测为
\begin{align*}
    \wv^\top \phi(\zv) & = (a_1 \phi(\xv_1) + a_2 \phi(\xv_2) + \cdots + a_m \phi(\xv_m))^\top \phi(\zv)     \\
                       & = a_1 \kappa(\xv_1, \zv) + a_2 \kappa(\xv_2, \zv) + \cdots + a_m \kappa(\xv_m, \zv)
\end{align*}

\section{实验}

如果是用matlab做实验,图省事可以直接调用quadprog函数,这是matlab优化工具箱自带的,但matlab的版本最好不要晚于2018,越新越好,最好是2020b,早期版本的matlab的优化工具箱很坑。如果没有新版本的matlab,可以用cvx工具箱(\url{http://cvxr.com/cvx/}),这是凸优化那本书的作者Stephen P. Boyd他们搞的,可以无缝嵌入到matlab代码里,很好用。

如果是用python做实验,可以用cvxopt包(\url{https://cvxopt.org/userguide/coneprog.html#quadratic-programming}),或者cvxpy包(\url{https://www.cvxpy.org/examples/basic/quadratic_program.html}),后者是matlab下的cvx工具箱的python移植版。

\end{document}

