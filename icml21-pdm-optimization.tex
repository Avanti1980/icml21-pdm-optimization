\documentclass{ctexart}
\usepackage{avanti}

\begin{document}
\title{PDM的优化}
\author{Teng Zhang}
\date{\today}
\maketitle

目前的优化问题为
\begin{align*}
    \min_{\wv} ~ \frac{1}{2m} \sum_{i \in [m]} \max \left\{ 0, 1 - \frac{y_i \wv^\top \xv_i}{\bar{\gamma}} \right\}^2, \quad \st ~ \bar{\gamma} > 0, ~ \| \wv \|_2 = 1
\end{align*}
这个形式是有问题的,两个约束都是多余的。注意间隔均值$\bar{\gamma} = \wv^\top \overline{y \xv}$关于$\wv$也是线性的,所以那个比值项是齐次的,放缩$\wv$不影响目标函数值。若求得的$\wv$使得$\bar{\gamma} < 0$,则$-\wv$可使得$\bar{\gamma} > 0$且不改变目标函数值、不破坏第二个约束。同理第二个约束也是多余的,若求得的$\wv$范数不为$1$,可以将其缩放使范数为$1$且不改变目标函数值。

\section{重新优化}

将两个约束都去掉,问题变成
\begin{align*}
    \min_{\wv} ~ \frac{1}{2m} \sum_{i \in [m]} \max \left\{ 0, 1 - \frac{y_i \wv^\top \xv_i}{\bar{\gamma}} \right\}^2
\end{align*}
由于放缩$\wv$不影响目标函数值,不妨放缩使$\bar{\gamma} = 1$,于是问题可进一步写成
\begin{align*}
    \min_{\wv} ~ \frac{1}{2m} \sum_{i \in [m]} \max \left\{ 0, 1 - y_i \wv^\top \xv_i \right\}^2, \quad \st ~ \sum_{i \in [m]} y_i \wv^\top \xv_i = m
\end{align*}
这个形式直接看也是合理的,最优情况是所有样本间隔均为$1$,这样损失为零;若有样本间隔$> 1$,由于间隔的和为$m$,必然存在另一个样本间隔$< 1$,这样就会产生损失。

记$\epsilon_i = \max \{ 0, 1 - y_i \wv^\top \xv_i \}$,则问题可写成
\begin{align*}
    \min_{\wv, \epsilon_i} ~ \frac{1}{2m} \sum_{i \in [m]} \epsilon_i^2, \quad \st ~ \epsilon_i \ge 0, ~ \epsilon_i \ge 1 - y_i \wv^\top \xv_i, ~ \sum_{i \in [m]} y_i \wv^\top \xv_i = m
\end{align*}
其中约束$\epsilon_i \ge 0$是多余的,因为若求得的$\epsilon_i < 0$,令其为零不破坏约束$\epsilon_i \ge 1 - y_i \wv^\top \xv_i$且可以使目标函数值变小,因此不存在这种情况。最终的问题为
\begin{align} \label{eq: primal}
    \min_{\wv, \epsilon_i} ~ \frac{1}{2m} \sum_{i \in [m]} \epsilon_i^2, \quad \st ~ \epsilon_i \ge 1 - y_i \wv^\top \xv_i, ~ \sum_{i \in [m]} y_i \wv^\top \xv_i = m
\end{align}
目标函数是二次的,约束都是线性的,因此这是一个凸二次规划,直接调用成熟的QP求解器即可。

记$\Xv = [\xv_1, \xv_2, \ldots, \xv_m] \in \Rbb^{d \times m}$,$\Yv = \diag \{ y_1, y_2, \ldots, y_m\}$,$\epsilonv = [\epsilon_1; \epsilon_2; \ldots; \epsilon_m]$,式(\ref{eq: primal})的矩阵形式为
\begin{align*}
    \min_{\wv, \epsilonv} ~ \frac{1}{2m} \epsilonv^\top \epsilonv, \quad \st ~ \epsilonv \ge \ev - \Yv \Xv^\top \wv, ~ \ev^\top \Yv \Xv^\top \wv = m
\end{align*}
记$\uv = [\wv; \epsilonv] \in \Rbb^{d+m}$,写成QP的标准形式是
\begin{align*}
    \min_{\uv} ~ \frac{1}{2} \uv^\top \begin{bmatrix}
        \zerov \\ & \frac{\Iv}{m}
    \end{bmatrix} \uv, \quad \st ~ [- \Yv \Xv^\top, -\Iv] \uv \le -\ev, ~ [\Xv \Yv \ev; \zerov]^\top \uv = m
\end{align*}
这样就可以直接掉包了。

对偶函数为
\begin{align*}
    L = \frac{1}{2m} \epsilonv^\top \epsilonv - \betav^\top (\epsilonv - \ev + \Yv \Xv^\top \wv) - \mu (\ev^\top \Yv \Xv^\top \wv - m)
\end{align*}
令$L$关于$\wv$的偏导数为零
\begin{align*}
    \frac{\partial L}{\partial \wv} = - \Xv \Yv \betav - \mu \Xv \Yv \ev = \zerov \Longrightarrow \Xv \Yv \betav + \mu \Xv \Yv \ev = \zerov
\end{align*}
令$L$关于$\epsilonv$的偏导数为零
\begin{align*}
    \frac{\partial L}{\partial \epsilonv} = \frac{\epsilonv}{m} - \betav = \zerov \Longrightarrow \epsilonv = m \betav
\end{align*}
故对偶问题为
\begin{align*}
    \max_{\betav, \mu} ~ - \frac{m}{2} \betav^\top \betav + \betav^\top \ev + \mu m, \quad \st ~ \Xv \Yv \betav + \mu \Xv \Yv \ev = \zerov, ~ \betav \ge \zerov
\end{align*}
记$\alphav = [\betav; \mu] \in \Rbb^{m+1}$,写成QP的标准形式是
\begin{align*}
    \min_{\alphav} ~ \frac{1}{2} \alphav^\top \begin{bmatrix}
        m \Iv \\ & 0
    \end{bmatrix} \alphav - [\ev; m]^\top \alphav, \quad \st ~ [\Xv \Yv, \Xv \Yv \ev] \alphav = \zerov, ~ \begin{bmatrix}
        - \Iv \\ & 0
    \end{bmatrix} \alphav \le \zerov
\end{align*}

原问题的变量个数为$d + m$,有$m$个线性不等式和$1$个线性等式约束;对偶问题的变量个数为$m+1$,有$d$个线性等式约束,不等式约束很简单可忽略,因此直接调包求解的话,可能对偶问题效率会稍微高点,但不绝对,跟$m$和$d$的值有关系。此外,由于两个问题都有大量形式复杂的约束,所以目前看不存在像SVM、ODM那样高效的基于坐标下降的解法。

\section{核化}

传统核方法都有$\wv$的二次项,再结合损失函数$y \wv^\top \xv \ge 1 - \epsilon$这样的一次项,对偶函数一求导就能得到最优的$\wv$是由$\xv_1, \ldots, \xv_m$线性表出的,即所谓的表示定理,这样回代到二次项中就出现了$\xv_i^\top \xv_j$的内积形式,之后就可以引入核函数了。

PDM目前的形式是无法直接核化的,因为目标函数中没有$\wv$的二次项,而间隔损失函数和$\wv$的二次项似乎不兼容(开头已述)。我目前想到是利用$\Kv \alphav = \zerov \Longleftrightarrow \Kv^\top\Kv \alphav = \zerov$这个结论,于是对偶问题中的等式约束可以改写为
\begin{align*}
    [\Xv \Yv, \Xv \Yv \ev]^\top [\Xv \Yv, \Xv \Yv \ev] \alphav = \zerov
\end{align*}
这样就有了$\Xv^\top \Xv$这样的形式,且$\Xv$只以$\Xv^\top \Xv$的形式出现,因此可以引入核函数。求得$\betav$和$\mu$后,根据$\epsilonv = m \betav$可以得到$\epsilonv$,利用互补松弛条件
\begin{align*}
    \beta_i (\epsilon_i - 1 + y_i \wv^\top \phi(\xv_i)) = 0
\end{align*}
对于大于零的$\beta_i$,可以求得$\wv^\top \phi(\xv_i) = y_i - y_i \epsilon_i = y_i - y_i m \beta_i$,这些都是有损失的样本,即“支持向量”。设所有的支持向量为$\xv_1, \xv_2, \ldots, \xv_t$,假如表示定理成立,设
\begin{align} \label{eq: span}
    \wv = a_1 \phi(\xv_1) + a_2 \phi(\xv_2) + \cdots + a_t \phi(\xv_t)
\end{align}
两边依次左乘$\phi(\xv_1)^\top, \ldots, \phi(\xv_t)^\top$可以得到$t$个方程,求解线性方程组可以得到$a_1, \ldots, a_t$,最后预测
\begin{align*}
    \wv^\top \phi(\zv) = (a_1 \phi(\xv_1) + a_2 \phi(\xv_2) + \cdots + a_t \phi(\xv_t))^\top \phi(\zv)
\end{align*}

所以最终问题就是表示定理是否成立,表示定理要求目标函数中的正则项$\Omega(\| \wv \|)$,其中$\Omega(\cdot)$是单调增函数,PDM没有正则项,可以理解为这里的$\Omega(\cdot)$恒为零函数,但零函数也是单调增函数,所以表示定理成立?

\end{document}

